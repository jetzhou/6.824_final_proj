\documentclass[12pt]{article}
\usepackage{fullpage}
%\usepackage{singlespace}

\title{Adding Permissions and Authentication to YFS \\ MIT 6.824 Project Report}
\author{Jeff Bezanson, Frances Zhang, Jet Zhou}

\begin{document}
\maketitle

\setcounter{secnumdepth}{1}

\subsubsection{Introduction}

The traditional Unix security model provides identification via user
and group names and IDs, and authentication via password login.
Access control is provided per-file, where each file
has a user, group, and permissions bits (corresponding to read, write,
and execute priviledges). This model was designed for use on mainframes,
where multiple users log in to a single machine, running a single
filesystem on a single kernel. Therefore the isolation between kernel
space and user space is sufficient to enforce security, since making
system calls is the only way to access filesystem data.

Networked, distributed filesystems change this situation dramatically.
Any client may attempt to connect to a network service, and we do not
have the benefit of the {\tt login} program to ensure that users are
who they say they are. Therefore, although we wish to interoperate with
existing Unix user domains, we must add an extra layer of authentication
to make sure attackers cannot break security simply by going around the
filesystem interface (i.e. connecting directly to the storage server
with forged credentials).

\subsubsection{Authentication and Signin}

One challenge of implementing a secure filesystem within the FUSE
architecture is that the interface was originally designed to be
part of a complete kernel environment, and not used in isolation from
its security context. For example, the {\tt read}
and {\tt write} calls do not tell us who is attempting the operation.
To solve this problem, we need users to sign in to the filesystem.
In our implementation this is done at mount time; each user must
create and use their own mount point accessing the shared filesystem.
Setting restrictive ({\tt 0700}) permissions on this directory keeps
the mount points private.

Users may supply a ``semi-secret'' key in the file {\tt \~{}/.yfs}, or
the system will generate one automatically on first use. This key may
be the public part of a cryptographic key pair, and is used to identify
a user to YFS. YFS internally associates these keys with Unix user IDs,
so a user can connect to the filesystem on any machine where they have
both the same user ID and their YFS key file available. YFS will
only accept extent server operations tagged with an ID and key that has
been seen before, and matches the stored permissions for the file in
question.

The key is ``semi-secret'' because if encryption is not used, then
security depends on this key staying secret. However, it is possible
for an attacker to intercept the key. To protect against this, the
key could be an RSA public key used to encrypt filesystem data. Then
attackers may be able to obtain the encrypted data, but only the user
who holds the corresponding private key will be able to decipher it.

\subsubsection{Groups}

Groups represent a further challenge, since group operations like
{\tt groupmod} and {\tt groupadd} only affect the Unix user database,
and our filesystem is not able to see them. To deal with this, our
extent server keeps its own copy of group information, which may become
out of sync with the user database. Initially, groups are empty except
that each user belongs to their own associated group. This way, with
no further information, our server behaves conservatively and rejects
operations that would only be allowed by group membership. We provide
a separate tool {\tt yfs\_usermod} for adding users to groups in YFS,
which is only supported for users YFS already knows about. It would be
possible for an administrator to run a script to pre-populate YFS
with user keys and all group memberships. In our experience, groups
and membership change quite infrequently, so this extra manual step
may be acceptable.

\end{document}
